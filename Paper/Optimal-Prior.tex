%%%%%%%%%%%%%%%%%%%%%%%%%%%%%%
%%%%%%%%%%%% Preamble %%%%%%%%%%%%%
%%%%%%%%%%%%%%%%%%%%%%%%%%%%%%

% Word target: 2,000 words (~ 1,981 currently)

% Outline:
% I. Introduction (300 words) (~307 currently and checked)
% II. GCE estimator (450 words) (~379 currently and checked)
% III. MP-GCE estimator (350 words) (~391 currently and checked) 
% IV. Sampling Experiments (450 words) (~353 currently and checked) 
% V. Results and Discussion (500 words) (~551 currently and checked)

% Declare document class and miscellaneous packages
\documentclass{elsarticle}
\usepackage{natbib}
\usepackage{amsmath}
\usepackage{mathrsfs}
\usepackage{amssymb}
\usepackage{ctable}
\usepackage{setspace}
\usepackage{longtable}
\usepackage{url}
\usepackage{moredefs,lips} 
\usepackage{IEEEtrantools}
\usepackage{multirow}
\usepackage{enumerate}
\usepackage[normalsize]{caption}
\usepackage{subcaption}
\usepackage{afterpage}
\usepackage[all]{nowidow}
\usepackage{listings}
\usepackage{graphicx}
\urlstyle{rm}

%Hyper-references
\usepackage{hyperref}
\hypersetup{colorlinks, citecolor=black, filecolor=black, linkcolor=black, 
urlcolor=black, pdftex}

% Begin document and frontmatter
\begin{document}
\begin{frontmatter}

% Title
\title{Incorporating Prior Information When True Priors are Unknown: An Information-Theoretic 
Approach for Increasing Efficiency in Estimation}

% Authors and affiliation
\author[hh]{Heath Henderson\corref{cauthor}}
\cortext[cauthor]{Corresponding author; 166D Heady Hall, Ames, IA 50011; 
Tel: +1 515 294 8122;  Email: \url{heathh@iastate.edu}.}
\author[ag]{Amos Golan}
\author[ss]{Skipper Seabold}

\address[hh]{Department of Economics, Iowa State University}
\address[ag]{Info-Metrics Institute and Department of Economics, 
American University}
\address[ss]{Department of Economics, American University}

% Abstract
\begin{abstract}
Prior information can improve inference, but in the social sciences observing such
information is quite rare.
We develop here a new way to incorporate prior information within an 
Information-Theoretic (IT) estimation framework. 
The estimator considers a wide range of potential priors and then uses a simple statistic 
to choose the optimal model. 
The optimal model is the one where the informational distance between a certain prior and 
the solution is minimized.
Through a large number of sampling experiments, we demonstrate the small sample 
performance gains of our method relative to its classical competitors.
\end{abstract}

% Keywords and JEL codes
\begin{keyword}
Entropy \sep Information \sep Maximum entropy \sep 
Generalized maximum entropy \sep Generalized cross entropy \sep Priors \\
\textit{JEL codes}: C13 \sep C14  
\end{keyword}

\end{frontmatter}

\doublespacing

%%%%%%%%%%%%%%%%%%%%%%%%%%%%%%
%%%%%%%%%%% Introduction %%%%%%%%%%%%
%%%%%%%%%%%%%%%%%%%%%%%%%%%%%%

\section{Introduction}
\label{sec: intro}

Information-theoretic (IT) methods of inference were developed as a means to
relax classical statistical assumptions in the estimation of both linear and 
non-linear models.
The generalized maximum entropy (GME) estimator is a leading IT method 
that assumes limited knowledge of statistical error structure and treats both
signal and noise as unknown quantities.
To retrieve the unknown quantities in an inherently under-determined problem, 
the GME estimator builds on maximum entropy formalism \citep{jaynes1957a,
 jaynes1957b} 
and uses Boltzmann-Shannon entropy \citep{shannon1948} as the objective 
functional.
Primarily in the context of ill-posed and ill-behaved problems, the GME 
estimator has been found to outperform its traditional counterparts 
(e.g.\ ordinary least squares and maximum likelihood) and has thus
witnessed widespread application in the social and natural sciences 
\citep{golan1996}. 

In the presence of prior information, the generalized cross entropy (GCE) 
estimator instead uses Kullback-Liebler divergence \citep{kullback1951} 
to minimize the informational ``distance'' between the posterior 
probabilities and the researcher's chosen priors.
While for uniform priors the GME and GCE estimators are identical,  
more informative reference distributions can improve performance of the 
GCE estimator relative to GME. 
However, prior information is often missing and achieving improved 
performance implies choosing among a possibly infinite number of priors.
We thus propose a general framework for selecting among alternative 
reference distributions by using (minimal) informational divergence as a 
metric to select among a wide range of priors within a GCE framework.
Through a large number of sampling experiments, we demonstrate the 
small sample performance gains of our method relative to GME and 
classical competitors.

In Section \ref{sec: gce} we provide a brief summary of GCE formalism and 
in Section \ref{sec: mp-gce} we outline our estimator. 
Section \ref{sec: mce} details the sampling experiments through which we 
compare the performance of our estimation strategy to the leading competing 
estimators. 
Finally, in Section \ref{sec: results} we describe our results and offer 
brief discussion.

%%%%%%%%%%%%%%%%%%%%%%%%%%%%%%
%%%%%%%%%% The GCE Estimator  %%%%%%%%%
%%%%%%%%%%%%%%%%%%%%%%%%%%%%%%

\section{Generalized Cross Entropy Estimator}
\label{sec: gce}

Consider the following linear regression model: 
\begin{equation}
\mathbf{y} = \mathbf{X\beta} + \mathbf{\varepsilon}
\end{equation}

\noindent
where $\mathbf{y}$ is a $T$-dimensional vector of observations on the 
dependent variable, $\mathbf{X}$ is a $T\times K$ matrix of exogenous
variables, $\mathbf{\beta}$ is a $K$-dimensional vector of unknown 
parameters, and $\mathbf{\varepsilon}$ is a $T$-dimensional vector of 
random errors with mean zero.%
\footnote{Vectors will be bolded and matrices will be capitalized throughout.}
Each $\mathbf{\beta}_k$ and $\mathbf{\varepsilon}_t$ in the GCE 
framework is typically viewed as the mean value of some well-defined 
random variable, which we denote as $\mathbf{z}_k$ and $\mathbf{v}_t$, 
respectively.
Accordingly, let $\mathbf{p}_k$ be an $M$-dimensional proper 
probability distribution defined on the support $\mathbf{z}_k$ such that 
$\mathbf{\beta}_k = \sum_m p_{km}z_{km} = \mathbf{z}_k' 
\mathbf{p}_k$ where the prime symbol denotes transpose.
Similarly, where $\mathbf{v}_t$ is symmetric about zero, let $\mathbf{w}_t$ 
be a $J$-dimensional proper probability distribution defined on $\mathbf{v}_t$ 
such that $\mathbf{\varepsilon}_t = \sum_j w_{tj}z_{tj} = \mathbf{v}_t' 
\mathbf{w}_t$.

The linear regression model can then be reparameterized as follows:
\begin{equation}
\mathbf{y} = \mathbf{X\beta} + \mathbf{\varepsilon} = 
\mathbf{X Z p} + \mathbf{V w}
\label{eq: reparm}
\end{equation}

\noindent
where, letting $\mathbf{z}=[\mathbf{z}_1' ~ \mathbf{z}_2' ~ \cdots ~ 
\mathbf{z}_K' ]'$ and $\mathbf{v}=[\mathbf{v}_1' ~ \mathbf{v}_2' ~ 
\cdots ~ \mathbf{v}_T' ]'$, 
$\mathbf{Z}= (\mathbf{I}_K \otimes \mathbf{1}_M')\mathbf{z}$ and
$\mathbf{V}= (\mathbf{I}_T \otimes \mathbf{1}_J')\mathbf{v}$
where $\mathbf{1}$ denotes a vector of ones, $\mathbf{I}$ is the 
identity matrix, and $\otimes$ is the Kronecker product.
Further, $\mathbf{p} = [\mathbf{p}_1' ~ \mathbf{p}_2' ~ \cdots ~ 
\mathbf{p}_K' ]'$ and $\mathbf{w} = [\mathbf{w}_1' ~ \mathbf{w}_2' 
~ \cdots ~ \mathbf{w}_T' ]'$.
The dimensions of $\mathbf{Z}$ and $\mathbf{V}$ are then 
$K \times KM$ and $T \times TJ$, respectively, and the dimensions of 
$\mathbf{p}$ and $\mathbf{w}$ are $KM \times 1$ and $TJ \times 1$, 
respectively.%
\footnote{While it is possible to construct unbounded 
and continuous supports \citep{golan2012}, for the sake of simplicity the above 
support spaces are constructed as discrete and bounded.} 
We then estimate $\beta$ with minimal assumptions; treating the errors as an additional 
set of unknown quantities.
This is an under-determined problem. 

Let $\mathbf{q}$ be a $KM$-dimensional vector of prior weights for the 
parameters $\mathbf{\beta}$ with prior mean $\mathbf{Zq}$.
Analogously, let $\mathbf{u}$ be a $TJ$-dimensional vector of prior weights 
for the disturbances $\mathbf{\varepsilon}$ with prior mean $\mathbf{Vu}$.
The GCE estimator then selects $\mathbf{p}$, $\mathbf{w}$ $\gg$ 
$\mathbf{0}$ to minimize 
\begin{equation}
I({\mathbf{p}, \mathbf{q}, \mathbf{w}, \mathbf{u}}) = 
\mathbf{p}' \ln (\mathbf{p}/\mathbf{q}) + 
\mathbf{w}' \ln (\mathbf{w}/\mathbf{u})
\label{eq: ce}
\end{equation}

\noindent
subject to
\begin{equation}
\mathbf{y} = \mathbf{X Z p} 
+ \mathbf{V w}
\label{eq: glm}
\end{equation}
\begin{equation}
\mathbf{1}_K = (\mathbf{I}_K \otimes \mathbf{1}_M')\mathbf{p}
\label{eq: pp1}
\end{equation}
\begin{equation}
\mathbf{1}_T = (\mathbf{I}_T \otimes \mathbf{1}_J')\mathbf{w}
\label{eq: pp2}
\end{equation}

\noindent
where Eqs.\ (\ref{eq: glm}) are the data constraints and Eqs.\ 
(\ref{eq: pp1})-(\ref{eq: pp2}) are proper probability constraints.
See \citet[Chap.\ 6]{golan1996} or \citet[Chap.\ 6]{golan2008} for analytic solutions, 
discussion of efficient techniques for computation of the GCE solutions via the 
unconstrained dual or concentrated version of the problem, and issues of inference.
Finally, note that with uniform priors the GCE estimator is identical to 
the GME estimator, which instead maximizes $H(\mathbf{p},\mathbf{w}) 
= - \mathbf{p}' \ln (\mathbf{p}) - \mathbf{w}' \ln (\mathbf{w})$.

%%%%%%%%%%%%%%%%%%%%%%%%%%%%%%
%%%%%%%% The MP-GCE Estimator  %%%%%%%%
%%%%%%%%%%%%%%%%%%%%%%%%%%%%%%

\section{Multi-Prior Generalized Cross Entropy Estimator}
\label{sec: mp-gce}

As with all problems of inference, a central question pertains to the choice of 
priors. 
However, for many applications across the social sciences prior information 
does not exist. 
While researchers typically select uniform priors in these situations, more 
informative priors can significantly improve small sample performance.
One way to overcome the problem empirically is to specify a set of all (or 
nearly all) possible priors and utilize an appropriate criterion to choose one of 
the solutions.
Accordingly, we propose using the (information) divergence-minimizing 
prior in an effort to enhance small sample performance of the GCE 
estimator. 
Henceforth, we refer to our multi-prior GCE estimator as MP-GCE. 

Given a set of priors, we need to choose a rule that selects the best prior that is 
consistent with our observed sample. 
Our MP-GCE estimator minimizes Eq.\ (\ref{eq: ce}) subject to Eqs.\ (\ref{eq: glm})-(\ref{eq: pp2}) 
for (potentially many) alternative priors and then selects the model for 
which $I(\cdot)$ is itself minimized. 
Stated differently, there are as many ``optimal'' solutions as priors and, following the logic 
of information theory, we choose the solution that minimizes the Kullback-Leibler criterion 
$I(\cdot)$. 
This is the solution where the \emph{a priori} unknown prior is closest to the observed 
data and, as such, the MP-GCE estimator identifies the ``best'' prior out of the complete 
set of potential priors. 
While computationally intensive, we demonstrate below that a relatively small 
number of priors is sufficient for the MP-GCE estimator to outperform its 
traditional counterparts.

The question remains as to which alternative reference distributions to 
incorporate.
Let $\mathbf{q}=[\mathbf{q}_1' ~ \mathbf{q}_2' ~ \cdots ~ \mathbf{q}_K' ]'$ where
$\mathbf{q}_k$ is an $M$-dimensional vector of prior weights for the parameter 
$\beta_k$.
For the sake of clarity, we focus on $\mathbf{q}_2$ and assume $\mathbf{q}_k$ to 
be uniform for $k=1,3,4,\ldots, K$.
In developing candidate priors, we define proper probabilities on the vector 
$\mathbf{z}_2$ such that across alternative choices of $\mathbf{q}_2$ the 
expectation $\mathbf{z}_2'\mathbf{q}_2$  yields values that appropriately
span the support space. 
Let $S$ represent all  $M!$ possible permutations of 
$s = [1^\rho ~ 2^\rho ~ \cdots ~ M^\rho]$.
Normalizing each element of $S$ by $\sum_{m=1}^M m^\rho$, it is 
readily verified that we have $M!$ candidate $M$-dimensional vectors for 
$\mathbf{q}_2$ that span the support space.%
\footnote{The parameter $\rho \in [-\infty, \infty]$ permits the researcher
to control the dispersion of the alternative prior means about the center of the 
support space.
Notably, as $\rho \to 0$ all priors approach uniform and as $|\rho| \to 
\infty$ the prior means approach the boundaries of the support space.}
Incorporating the uniform prior $[1/M ~ 1/M ~ \cdots ~ 1/M]$, we have a total
of $M! + 1$ priors.
The MP-GCE estimator in this case minimizes Eq.\ (\ref{eq: ce}) subject 
to Eqs.\ (\ref{eq: glm})-(\ref{eq: pp2}) after inserting into $\mathbf{q}$ 
each of the $M! + 1$ priors.

%%%%%%%%%%%%%%%%%%%%%%%%%%%%%%
%%%%%%%%% Sampling Experiments %%%%%%%%%%
%%%%%%%%%%%%%%%%%%%%%%%%%%%%%%

\section{Sampling Experiments}
\label{sec: mce}

For each replication in the sampling experiments our objective is to minimize 
$I(\cdot)$ subject to the data consistency and proper probability constraints 
for alternative choices of $\mathbf{q}$, the prior weights for the parameter 
$\mathbf{\beta}$. 
To measure the performance of each prior we focus on the mean squared 
error MSE$(\hat{\mathbf{\beta}}) = E \| \hat{\mathbf{\beta}}- 
\mathbf{\beta}\|^2$.
To establish a basis for selecting among alternative priors, the relationship 
between MSE$(\hat{\mathbf{\beta}})$ and $E[I(\cdot)]$ is examined.
This relationship is considered for different sample sizes, levels of noise, 
and degrees of correlation among covariates.
Further, the performance of our MP-GCE estimator is referenced 
against the OLS and GME estimators.

Consider a baseline scenario where $\mathbf{X}$ in Eq.\ (\ref{eq: reparm})
is a $10 \times 3$ design matrix where $\mathbf{x}_k \sim 
N(0,1)$, $\mathbf{\beta}= [\beta_1 ~ \beta_2 ~ \beta_3]'=[1 ~ -3 ~ 2]'$, 
and $\mathbf{\varepsilon} \sim N(0,1)$.
In order to control correlation among covariates, the condition
number $\kappa(\mathbf{X}'\mathbf{X})=1$ is specified \textit{a priori}
by appropriately replacing the eigenvalues in the singular value 
decomposition of $\mathbf{X}$.%
\footnote{See \citet[pg.\ 133]{golan1996} and references therein for 
details.}
For $M=J=3$ points, the support vector for $\mathbf{p}_k$ is 
$\mathbf{z}_k = [- 10 ~ 0 ~ 10]'$ and the support vector for $\mathbf{w}_t$ 
is $\mathbf{v}_t = [- 3\sigma_y ~ 0 ~ 3\sigma_y]'$ where $\sigma_y$ is the sample
standard deviation of $\mathbf{y}$.%
\footnote{The error support vector is based on the three-sigma
rule \citep{pukelsheim1994} and is calculated 
uniquely for each replication based on the observed $\mathbf{y}$.}
Let $\mathbf{q} = [\mathbf{q}_1' ~ 
\mathbf{q}_2' ~ \mathbf{q}_3']'$ represent prior information on the 
respective elements of the parameter vector.
As in Section \ref{sec: mp-gce}, we let $\mathbf{q}_1 = \mathbf{q}_3 = 1/M 
= 1/3$ (i.e.\ uniform) and estimate the model using $\rho=0.5$  in the 
discussed alternative specifications of $\mathbf{q}_2$ denoted in Table \ref{tbl: priors}.
%\footnote{That is, the focus here is on examining alternative reference 
%distributions for the parameter $\beta_2$. 
%However, with respect to the performance of the estimator, we are 
%interested in the MSE across all parameters.
%As will be seen, with correlation among covariates the specification of the 
%prior on $\beta_2$ affects the estimation of other parameters.}
We use uniform priors on the error term throughout (i.e.\ $\mathbf{u} = 1/J = 1/3$).
Finally, the experiment is conducted with $N=1,000$ replications.

% Priors
\footnotesize
\ctable[
cap = {Priors for Experiments},
caption = {Alternative priors used in sampling experiments.\ $\mathbf{q}_2$ 
denotes the vector of priors associated with the parameter $\beta_2$.},
captionskip = -2ex,
pos=htb,
label = {tbl: priors}
]{lccc}{
}{\hline \hline
Label & $\mathbf{q_2}$  \\ \hline
1 & $[\sqrt{1} ~ \sqrt{2} ~ \sqrt{3}]'/\sum_{m=1}^3 \sqrt{m}$\\    
2 & $[\sqrt{1} ~ \sqrt{3} ~ \sqrt{2}]'/\sum_{m=1}^3 \sqrt{m}$ \\
3 & $[\sqrt{2} ~ \sqrt{1} ~ \sqrt{3}]'/\sum_{m=1}^3 \sqrt{m}$\\
4 & $[\sqrt{2} ~ \sqrt{3} ~ \sqrt{1}]'/\sum_{m=1}^3 \sqrt{m}$\\
5 & $[\sqrt{3} ~ \sqrt{1} ~ \sqrt{2}]'/\sum_{m=1}^3 \sqrt{m}$\\
6 & $[\sqrt{3} ~ \sqrt{2} ~ \sqrt{1}]'/\sum_{m=1}^3 \sqrt{m}$\\
7 & $[\sqrt{1/3} ~ \sqrt{1/3} ~ \sqrt{1/3}]/\sum_{m=1}^3 \sqrt{m}$ \\\hline}
\normalsize

We vary the sample sizes, the level of noise, and the degree of correlation among 
covariates in the experiments.
Beyond $T=10$ we consider sample sizes of $T=20, 50, 100, \text{and } 200$. 
We increase the level of noise in one set of experiments to  $\mathbf{\varepsilon} \sim 
N(0,5)$.
We also let $\kappa(\mathbf{X}'\mathbf{X})=100$, which represents a 
moderately ill-conditioned design matrix.
This gives a total of 15 experimental specifications as summarized in Table \ref{tbl: exp}.

% Sampling Experiments
\footnotesize
\ctable[
cap = {Sampling Experiments},
caption = {Summary of sampling experiments conducted.\  The columns marked $T$, 
$\varepsilon$, and $\kappa( \mathbf{X'X})$ display
the sample size, error specification, and condition number, respectively.},
captionskip = -2ex,
pos=htb,
label = {tbl: exp}
]{lccc}{
}{\hline \hline
Experiment & \multicolumn{1}{c}{$T$} & 
\multicolumn{1}{c}{$\varepsilon$} &
\multicolumn{1}{c}{$\kappa(\mathbf{X}'\mathbf{X})$} \\ \hline
 1 & 10 & $N(0,1)$ &   1   \\ 
 2 & 20 & $N(0,1)$ &   1    \\ 
 3 & 50 & $N(0,1)$ &   1    \\ 
 4 & 100 & $N(0,1)$ &   1    \\ 
 5 & 200 & $N(0,1)$ &   1    \\ 
6 & 10 & $N(0,5)$ &   1   \\ 
7 & 20 & $N(0,5)$ &   1   \\
8 & 50 & $N(0,5)$ &   1   \\  
9 & 100 & $N(0,5)$ &   1   \\
10 & 200 & $N(0,5)$ &   1   \\  
11 & 10 & $N(0,1)$ & 100   \\
12 & 20 & $N(0,1)$ & 100   \\ 
13 & 50 & $N(0,1)$ & 100   \\ 
14 & 100 & $N(0,1)$ & 100   \\ 
15 & 200 & $N(0,1)$ & 100   \\    \hline}
\normalsize

%%%%%%%%%%%%%%%%%%%%%%%%%%%%%%
%%%%%%%%%%%% Results %%%%%%%%%%%%%%
%%%%%%%%%%%%%%%%%%%%%%%%%%%%%%

\section{Results and Discussion}
\label{sec: results}

Figures \ref{fig: B1}-\ref{fig: B} and Tables \ref{tbl: mseb1}-\ref{tbl: mseb} 
present the results of the sampling experiments.
We first focus on the results associated with the estimation of $\beta_2$, which 
is the parameter for which we consider alternative priors.
For the baseline scenario, Figure \ref{fig: B1} plots the replication-average of 
$I(\mathbf{p_2}, \mathbf{q_2}) = \mathbf{p_2}'\ln (\mathbf{p_2}/
\mathbf{q_2})$ (i.e.\ the cross entropy associated with $\beta_2$) on 
MSE$(\hat{\beta}_2)$  for alternative choices of $\mathbf{q_2}$.
Table \ref{tbl: priors} provides the priors that correspond to each label.
Figure \ref{fig: B1} depicts a core result: we find a strong positive 
relationship between the average of $I(\mathbf{p_2}, \mathbf{q_2})$ 
and MSE$(\beta_2)$.
Notably, the estimator using the prior for which $I(\mathbf{p_2}, 
\mathbf{q_2})$ is minimized (i.e.\ prior ``6'') outperforms the GME 
estimator (i.e.\ prior ``7'') in terms of mean squared error loss.

% B1 figure
\begin{figure}[htbp]
\centering
\includegraphics[width=0.75\textwidth]{B2.pdf}
\caption{Cross entropy $I(\mathbf{p_2}, \mathbf{q_2})$ vs.\  the mean squared
error of $\hat{\beta}_2$ for the alternative priors used in the baseline experiment.
Prior labels are provided in Table \ref{tbl: priors}.
The strong positive relationship suggests using $I(\mathbf{p_2}, \mathbf{q_2})$
as a criterion for prior selection.}
\label{fig: B1}
\end{figure}

Examining the above in greater detail, the first experiment in Table 
\ref{tbl: mseb1} displays baseline results for three alternative estimators: 
(1) our MP-GCE estimator; (2) the GME estimator; and (3) the OLS 
estimator.
The MP-GCE and GME columns present the MSE$(\beta_2)$ measures 
depicted in Figure \ref{fig: B1} (i.e.\ for priors ``6'' and ``7'', respectively).
It is evident from Table \ref{tbl: mseb1} that GME outperforms OLS in 
terms of mean squared error loss, which is a well-known result 
\citep{golan1996}. 
Further, we see that our MP-GCE estimator outperforms both GME and OLS,
and the reduction in precision risk (i.e.\ MSE) is non-negligible.
For example, relative to GME, the MP-GCE estimator yields a 17 percent 
reduction in mean squared error loss.
These relationships are maintained in experiments 2-5 and we see that 
the magnitude of the precision risk reductions decreases in $T$.

% Mean squared error (B1)
\footnotesize
\ctable[
cap = {Mean Squared Error $\hat{\beta}_2$},
caption = {Mean Squared Error of $\hat{\beta}_2$ for Alternative 
Experiments and Estimators},
captionskip = -2ex,
pos=htb,
label = {tbl: mseb1}
]{lccc}{
}{\hline \hline
Experiment & \multicolumn{1}{c}{MP-GCE} & 
\multicolumn{1}{c}{GME} &
\multicolumn{1}{c}{OLS}  \\ \hline
1 & 0.69 & 0.83  & 1.03  \\ 
2 & 0.70 & 0.78 & 0.93 \\ 
3 & 0.78 & 0.83 & 0.95  \\ 
4 & 0.86 & 0.90 & 0.96   \\ 
5 & 0.94  & 0.97  & 0.98  \\ 
6 & 3.64 & 7.06  & 25.67 \\ 
7 & 3.49 & 6.93  & 23.32 \\
8 & 4.28 & 7.59  & 23.71 \\  
9 & 8.09 & 9.57 & 24.04 \\
10 & 10.63 & 11.77 & 24.50  \\  
11 & 5.47 & 8.16  & 35.66 \\
12 & 1.35 & 1.96 & 8.15 \\ 
13 & 0.48 & 0.49 & 0.55  \\ 
14 & 0.94 & 1.05  & 2.37  \\ 
15 & 1.29 & 1.33 & 1.72 \\     \hline}
\normalsize

The largest reductions in precision risk are, however, associated with the noisy
and ill-conditioned experiments.
Rows 6-10 in Table \ref{tbl: mseb1} present the experiments with increased 
noise. 
Looking to experiment 6, which is the noisy case with $T=10$, we see that 
the GME estimator leads to a 72 percent reduction in precision risk relative 
to OLS.
Further, relative to GME, the MP-GCE estimator reduces precision risk by 
approximately 48 percent. 
As is evident from experiments 7-10, we again see convergence in mean 
squared error as sample size increases.
Similar results hold for the ill-conditioned cases presented in rows 11-15.
For the $T=10$ case in row 11, we see that GME outperforms OLS 
(a 77 percent reduction in precision risk) and MP-GCE outperforms GME 
(a 33 percent reduction in precision risk).
Relative to the noisy case, however, the precision risk reductions for the 
ill-conditioned case appear to dissipate faster as sample size increases.

Figure \ref{fig: B} depicts a second core result: we find a strong positive 
relationship between the replication-average of $I(\mathbf{p}, \mathbf{q}) = 
\mathbf{p}'\ln (\mathbf{p}/\mathbf{q})$ (i.e.\ the cross entropy of the signal) 
and MSE$(\beta)$.
Further, Table \ref{tbl: mseb} presents MSE$(\beta)$ for all experiments and 
estimators, and displays results qualitatively identical to those discussed above:
GME outperforms OLS and MP-GCE consistently outperforms GME.
Given such qualitative similarity we do not provide detailed discussion, 
nevertheless it is important to emphasize that these results suggest that our 
MP-GCE estimator may be used quite generally (i.e.\ with combinations of 
different priors on all parameters).
In fact, in additional experiments (available upon request) we incorporated 
alternative priors on $\beta_3$ and observed further performance gains,
particularly for the ill-conditioned case.
We thus conclude that our MP-GCE estimator provides a simple, 
computationally inexpensive, and flexible means to improve small sample 
performance, particularly for small, noisy, or ill-conditioned problems.

% B figure
\begin{figure}[htbp]
\centering
\includegraphics[width=0.75\textwidth]{B.pdf}
\caption{Cross entropy $I(\mathbf{p}, \mathbf{q})$ vs.\  the mean squared
error of $\hat{\beta}$ for the alternative priors used in the baseline experiment.
Prior labels are provided in Table \ref{tbl: priors}.
The strong positive relationship suggests using $I(\mathbf{p}, \mathbf{q})$
as a criterion for prior selection.}
\label{fig: B}
\end{figure}

% Mean squared error (B)
\footnotesize
\ctable[
cap = {Mean Squared Error $\hat{\beta}$},
caption = {Mean Squared Error of $\hat{\beta}$ for Alternative 
Experiments and Estimators},
captionskip = -2ex,
pos=htb,
label = {tbl: mseb}
]{lccc}{
}{\hline \hline
Experiment & \multicolumn{1}{c}{MP-GCE} & 
\multicolumn{1}{c}{GME} &
\multicolumn{1}{c}{OLS}  \\ \hline
1 & 2.13 & 2.27 & 2.98 \\ 
2 & 2.26 & 2.34 & 2.92 \\ 
3 & 2.46 & 2.52 & 2.93  \\ 
4 & 2.70 & 2.74 & 3.01   \\ 
5 & 2.88 & 2.89 & 2.99   \\ 
6 & 11.94 & 15.36 & 74.39 \\ 
7 & 11.81 & 15.26 & 72.94 \\
8 & 14.54 & 17.84 & 73.24 \\  
9 & 20.81 & 22.69 & 75.18  \\
10 & 24.77 & 29.47 &  74.71 \\  
11 & 12.72 & 12.72 & 52.60 \\
12 & 7.44 & 10.28 & 51.78 \\ 
13 & 5.23 & 5.22 & 51.99 \\ 
14 & 7.75 & 8.68 & 51.28  \\ 
15 & 7.52 & 8.00 & 53.76  \\     \hline}
\normalsize

%We now turn to the relationship between the replication-average of 
%$I(\mathbf{p}, \mathbf{q}) = \mathbf{p}'\ln (\mathbf{p}/
%\mathbf{q})$ (i.e.\ the cross entropy of the signal) and 
%MSE$(\hat{\mathbf{\beta}})$ for alternative choices of $\mathbf{q_2}$.%
%\footnote{Note here that we focus on the cross entropy of the signal 
%$I(\mathbf{p}, \mathbf{q})$ rather than the overall cross entropy 
%$I(\mathbf{p}, \mathbf{q}, \mathbf{w}, \mathbf{u})$.
%Cross entropy is an additive measure and as sample size increases the cross
%entropy of the noise $I(\mathbf{w}, \mathbf{u})$ will come to dominate
%that of the signal.
%As such, we find that the cross entropy of the signal provides a more reliable
%basis for selecting among alternative priors.}
%For the baseline scenario, Figure \ref{fig: B} plots these two quantities for 
%alternative prior choices and demonstrates a second core result: we find a 
%strong positive relationship between the average of $I(\mathbf{p}, 
%\mathbf{q})$ and MSE$(\hat{\mathbf{\beta}})$.
%Once again, the MP-GCE estimator (i.e.\ prior ``6'') outperforms the 
%GME estimator (i.e.\ prior ``7''), and the first row of Table 
%\ref{tbl: mseb} implies a precision risk reduction of approximately 6 
%percent.
%Both estimators again outperform OLS and it is evident from 
%experiments 2-5 that all estimators converge in mean squared error as 
%the sample size increases.

%Looking to the noisy cases presented in rows 6-10 in Table 
%\ref{tbl: mseb}, we see once more that the mean squared error 
%reductions are larger than the baseline case. 
%Examining experiment 6, which is the noisy case with $T=10$, it is 
%evident that GME leads to a precision risk reduction of 79 percent
%relative to OLS. 
%Further, the MP-GCE estimator leads to precision risk reduction of 
%approximately 22 percent relative to GME.
%These reductions again dissipate as sample size increases, as is evident 
%from experiments 7-10.
%Finally, looking to the ill-conditioned cases in experiments 11-15, we see 
%that GME continues to outperform OLS and MP-GCE generally outperforms
%GME. 
%While in experiment 11 the performances of MP-GCE and GME are indeed
%identical, we do see some performance gains in other cases.
%For example, experiments 12 and 14 display 28 and 11 percent reductions 
%in precision risk, respectively.

%We conclude this section with a brief discussion of additional sampling 
%experiments conducted. 
%First, we increased the number of regressors from 3 to 7 and did not observe a 
%qualitative change in our results in any case.
%Second, we incorporated alternative priors on $\beta_3$ and found only 
%modest precision risk reductions for the baseline and noisy cases, but 
%non-negligible reductions in the ill-conditioned case (e.g.\ MP-GCE witnessed 
%MSE$(\beta)=7.85$ for $T=10$).%
%\footnote{In combination with the $M! + 1$ priors on $\beta_2$, here we 
%considered the same $M! + 1$ priors on $\beta_3$.
%Thus, for $M=3$, these experiments used a total of $(M! + 1)\times (M! + 1) 
%= 49$ alternative reference distributions.}
%Finally, we doubled the parameter vector, which served to generate a situation 
%where the parameter support was moderately ill-specified.
%While MP-GCE continued to outperform GME in this case, OLS outperformed 
%both in select cases, most notably for the baseline specification.
%This final experiment underscores the importance of appropriately specifying
%the parameter support vector.

%%%%%%%%%%%%%%%%%%%%%%%%%%%%%%
%%%%%%%%%%% Conclusions %%%%%%%%%%%%%
%%%%%%%%%%%%%%%%%%%%%%%%%%%%%%

\section{Conclusions}
\label{sec: conc}

We proposed a new approach for incorporating a whole class of priors when the true 
priors are unknown. 
Incorporating this within an IT framework, we have shown that our method outperforms 
its competitors for all finite data. 
This new approach is easy to use and apply, and can be used within linear or 
non-linear IT methods. 
Since this method is a generalization of the GCE estimator, all inferential statistics and 
large sample properties are already well known. 
We thus concentrated here on the increase in small sample efficiency.

%%%%%%%%%%%%%%%%%%%%%%%%%%%%%%
%%%%%%%%%%%% References %%%%%%%%%%%%
%%%%%%%%%%%%%%%%%%%%%%%%%%%%%%

%Start fresh page
\newpage
\cleardoublepage
\singlespacing

%Declare the style and file to use
\section*{References}
\bibliographystyle{model5-names}\biboptions{authoryear}
\bibliography{/Users/hendersonhl/Documents/References}

\end{document}